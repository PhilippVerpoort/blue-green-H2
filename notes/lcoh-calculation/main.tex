%% ****** Start of file aapmtemplate.tex ****** %
%%
%%   This file is part of the files in the distribution of AAPM substyles for REVTeX4.2.
%%   Version 4.2a of January 28, 2015.
%%
%
% This is a template for producing documents for use with 
% the REVTEX 4.2 document class and the AAPM substyles.
% 
% Copy this file to another name and then work on that file.
% That way, you always have this original template file to use.

\documentclass[%
 draft,
 aapm,
 mph,%
 amsmath,amssymb,
%preprint,%
 reprint,%
%author-year,%
%author-numerical,%
]{revtex4-2}

\usepackage[utf8]{inputenc}

\usepackage{amsmath}
\usepackage{amssymb}
\usepackage{mathtools}
\usepackage{upgreek}

\usepackage{siunitx}

\usepackage{mhchem}

\usepackage{xspace}

% Defining string as labels of certain blocks.

\begin{document}

% Use the \preprint command to place your local institutional report number 
% on the title page in preprint mode.
% Multiple \preprint commands are allowed.
%\preprint{}

\title{Calculating LCOH -- Deriving Equations} %Title of paper

% repeat the \author .. \affiliation  etc. as needed
% \email, \thanks, \homepage, \altaffiliation all apply to the current author.
% Explanatory text should go in the []'s, 
% actual e-mail address or url should go in the {}'s for \email and \homepage.
% Please use the appropriate macro for the type of information

% \affiliation command applies to all authors since the last \affiliation command. 
% The \affiliation command should follow the other information.

\author{Philipp C. Verpoort}
%\email[]{Your e-mail address}
%\homepage[]{Your web page}
%\thanks{}
%\altaffiliation{}

% Collaboration name, if desired (requires use of superscriptaddress option in \documentclass). 
% \noaffiliation is required (may also be used with the \author command).
%\collaboration{}
%\noaffiliation

%\date{\today}

%% -----------------------
%% |      References     |
%% -----------------------
\newcommand{\secref}[2][]{Sec.~\ref{sec:#2}#1}
\newcommand{\appref}[2][]{App.~\ref{sec:#2}#1}
\newcommand{\chapref}[2][]{Chap.~\ref{chap:#2}#1}
\newcommand{\figref}[2][]{Fig.~\ref{fig:#2}#1}
\newcommand{\tabref}[2][]{Tab.~\ref{tab:#2}#1}
\newcommand{\eqnref}[2][]{Eq.~(\ref{eq:#2}#1)}

\newcommand{\subfig}[1]{{\bf(#1)}}


%% -----------------------
%% |        maths        |
%% -----------------------
\newcommand{\arr}[2]{\begin{array}{#1}#2\end{array}}
\newcommand{\op}[1]{\operatorname{#1}}
\newcommand{\punc}[1]{\,#1}
\newcommand{\anglemean}[1]{\left\langle#1\right\rangle}

\renewcommand{\d}{\ensuremath{ \mathrm{d} }}
\newcommand{\dd}[1]{\ensuremath{ \mathrm{d}#1 }}
\newcommand{\ddd}[2]{\ensuremath{ \mathrm{d}^{#2}\!{#1} }}
\newcommand{\pdb}[2]{\frac{\partial #1}{\partial #2}}
\newcommand{\dbyd}[3][{}]{\frac{\mathrm{d}^{#1}#2}{\mathrm{d}{#3}^{#1}}}
\newcommand{\dbydsecond}[3]{\frac{\mathrm{d}^2#1}{\mathrm{d}#2\,\mathrm{d}#3}}


%% -----------------------
%% |       various       |
%% -----------------------
\newcommand{\prc}{p_{\mathrm{c}}}
\newcommand{\prYX}{p_{Y\rightarrow X}}
\newcommand{\prg}{p_{r\rightarrow g}}
\newcommand{\prb}{p_{r\rightarrow b}}
\newcommand{\prbg}{p_{b\rightarrow g}}



%% -----------------------
%% |      molecules      |
%% -----------------------
\newcommand{\hydrogen}{\ce{H2}\xspace}
\newcommand{\carbdiox}{\ce{CO2}\xspace}
\newcommand{\carbmono}{\ce{CO}\xspace}
\newcommand{\ammonia}{\ce{NH3}\xspace}
\newcommand{\methane}{\ce{CH4}\xspace}
\newcommand{\methanol}{\ce{MeOH}\xspace}
\newcommand{\ethanol}{\ce{EtOH}\xspace}
\newcommand{\nitrogen}{\ce{N2}\xspace}



%% -----------------------
%% |         LCOH        |
%% -----------------------
\newcommand{\Ppl}{P_{\mathrm{pl}}}
\newcommand{\cpl}{C_{\mathrm{pl}}}
\newcommand{\pf}{p_{\mathrm{f}}}
\newcommand{\cing}{ci_{\mathrm{ng}}}
\newcommand{\cCTS}{c_{\mathrm{CTS}}}
\newcommand{\LCOHb}{LCOH_{\mathrm{blue}}}

\newcommand{\cL}{c_{\mathrm{L}}}
\newcommand{\pel}{p_{\mathrm{el}}}
\newcommand{\LCOHg}{LCOH_{\mathrm{green}}}


\maketitle %\maketitle must follow title, authors, abstract

% Body of paper goes here. Use proper sectioning commands. 
% References should be done using the \cite, \ref, and \label commands
\section{Methods for determining LCOH}

Info on how to determine LCOE from CAPEX, OPEX, lifetime, irate, etc can be taken from the LCOE report by Fraunhofer ISE~\cite{lcoe_fraunhofer_ise}.

Let us assume the following variables:

\setlength{\tabcolsep}{0.8em}
\begin{tabular}{>{$}l<{$}l}
I_0 & Investment expenditure (EUR)\\
A_t & Annual operational cost in year t (EUR)\\
M_t & Produced amount of \ce{H2} (kWh)\\
i   & Real interest rate / WACC (\%)\\
n   & Economic lifetime in years\\
t   & Year of lifetime $(1,2,\ldots,n)$
\end{tabular}
We assume that the annually produced amounts of \ce{H2} and annual operational cost are constant with time such that $M_t =: M$ and $A_t =: A$.

In the net-present value method (NPV), we can write
\begin{align}
    LCOH &= \frac{I_0 + \sum_{t=1}^{n} \frac{A_t}{(1+i)^t}}{\sum_{t=1}^{n} \frac{M_t}{(1+i)^t}}\\
         &= \frac{I_0 + A \sum_{t=1}^{n} \frac{1}{(1+i)^t}}{M \sum_{t=1}^{n} \frac{1}{(1+i)^t}}\\
         &= \frac{\epsilon \, I_0 + A}{M} \qquad \text{with} \quad \epsilon = \frac{1}{\sum_{t=1}^{n} \frac{1}{(1+i)^t}} \punc.
\end{align}
Alternatively, we can use the so-called annuity method to directly write
\begin{align}
    LCOH &= \frac{\epsilon \, I_0 + A}{M} \qquad \text{with} \quad \epsilon = \frac{i(1+i)^n}{(1+i)^n-1} \punc. \label{eq:lcoh_economics}
\end{align}
with the fixed charge rate (or `annuity factor' (ANF)). Turns out after some maths that these two are the same:
\begin{align}
    \sum_{t=1}^{n} \frac{1}{(1+i)^t} &= \sum_{t=0}^{n} \left(\frac{1}{1+i}\right)^t - 1 \\
                                     &= \frac{1-\left(\frac{1}{1+i}\right)^{n+1}}{1-\left(\frac{1}{1+i}\right)} - 1 \\
                                     &= \frac{(1+i)^{n+1} - 1}{(1+i)^{n+1} - (1+i)^n} - 1 \\
                                     &= \frac{(1+i)^n - 1}{(1+i)^{n+1} - (1+i)^n} \\
                                     &= \frac{(1+i)^n - 1}{(1+i)^n (1+i-1)} \\
                                     &= \left( \frac{i \, (1+i)^n}{(1+i)^n - 1} \right)^{-1}.
\end{align}
So ultimately, we can always just use \eqnref{lcoh_economics}:
\begin{equation}
    LCOH = \underbrace{\frac{\epsilon \, I_0}{M}}_{CAPEX} + \underbrace{\frac{A}{M}}_{OPEX} \quad \text{with} \quad \epsilon = \frac{i(1+i)^n}{(1+i)^n-1} \punc.
\end{equation}


\section{Green hydrogen}
OK cool. Let's see what this looks like for green hydrogen. In this case, we can write the CAPEX term as

\begin{equation}
    \epsilon \frac{I_0}{M} = \frac{\epsilon\,\cL}{\SI{8760}{\hour} \cdot OCF} \punc,
\end{equation}
where $OCF$ is the Operational Capacity Factor, and $\cL$ is the electrolyser cost in units of \si{EUR\per\kWh}. Using the electricity price $\pel$, the annual cost can be written as
\begin{equation}
    A  = \pel \frac{M}{\varphi} \punc.
\end{equation}
Hence, the LCOH is given as
\begin{equation}
    \LCOHg = \frac{\epsilon\,\cL}{\SI{8760}{\hour} \cdot OCF} + \frac{\pel}{\varphi} \punc.
\end{equation}
This is rather simple, so we extend this equation in the following way:
\begin{equation}
    \LCOHg = \alpha \frac{\epsilon\,\cL}{\SI{8760}{\hour} \cdot OCF^\theta} + \gamma \frac{\pel}{\varphi} \punc,
\end{equation}
where now $\alpha$, $\gamma$, and $\theta$ are dimensionless fitting coefficients. We can also simplify this equation by using the linearised approximation $OCF^\theta \approx OCF(1+(\theta-1)\log(OCF))$. Using $i = \SI{8}{\percent}$, $n = 25$, and $\varphi=0.7$, we obtain the following coefficients from linear regression:

\begin{table}[h!]
\begin{tabular}{|>{$}l<{$}||r|r|r|}
\hline
$Coeff.$ & SMR      \\ \hline\hline
\alpha   & 3.212    \\ \hline
\gamma   & 1.236    \\ \hline
\theta   & 1.009    \\ \hline
\end{tabular}
\end{table}


\section{Blue hydrogen}
Now, let's look at blue \ce{H2}. The investment expenditure equals to the plant cost $\cpl$. The production power of the plant is $\Ppl$, and hence the amount of production $M$ per year is $M = P_p \cdot \SI{8760}{\hour}$. We can therefore write the CAPEX term as
\begin{equation}
    \epsilon \frac{I_0}{M} = \frac{\epsilon\,\cpl}{\Ppl \cdot \SI{8760}{\hour}} \punc.
\end{equation}
The OPEX term consists of two contributions: fuel cost and Carbon Transport and Storage (CTS) cost. Using the fuel price $\pf$, the conversion efficiency $\varphi$, the carbon intensity of natural gas $\cing$, and the capture rate $CR$, and the CTS cost $\cCTS$, we can write the annual cost as
\begin{equation}
    A = \pf \cdot \frac{M}{\varphi} + \cCTS \cdot CR \,\cdot \cing \cdot \frac{M}{\varphi} \punc.
\end{equation}
The LCOH of blue hydrogen therefore becomes
\begin{equation}
    \LCOHb = \frac{\epsilon \, \cpl}{\Ppl \cdot \SI{8760}{\hour}} + \frac{\pf}{\varphi} + \frac{\cCTS \cdot CR \,\cdot \cing}{\varphi} \punc.
\end{equation}
This is rather simple, so we extend this equation in the following way:
\begin{equation}
    \LCOHb = \alpha \frac{\epsilon \, \cpl}{\Ppl \cdot \SI{8760}{\hour}} + \gamma \frac{\pf}{\varphi} + \mu\frac{\cCTS \cdot CR \,\cdot \cing}{\varphi} \punc,
\end{equation}
where now $\alpha$, $\gamma$, and $\mu$ are dimensionless fitting coefficients. We can also rewrite this using the coefficient $\mu'$ as
\begin{equation}
    \LCOHb = \alpha \frac{\epsilon \, \cpl}{\Ppl \cdot \SI{8760}{\hour}} + \gamma \frac{\pf}{\varphi} + \mu'\cCTS\,e \punc,
\end{equation}
where $e$ is now the emissions of the plant in units of $\si{\tonne_{\ce{CO2}eq}\per\kWh_{\ce{H2}}}$. Using $i = \SI{8}{\percent}$, $n = 25$, and $\cing = \SI{0.198}{\tonne_{\ce{CO2}eq}\per\kWh}$, we obtain the following coefficients from linear regression:

\begin{table}[h!]
\begin{tabular}{|>{$}l<{$}||r|r|r|}
\hline
$Coeff.$ & SMR   & HEB   & LEB   \\ \hline\hline
\alpha   & 3.474 & 4.619 & 4.622 \\ \hline
\gamma   & 3.000 & 3.019 & 3.028 \\ \hline
\mu      & --~~~ & 2.124 & 2.138 \\ \hline
\mu'     & --~~~ & 2.200 & 2.212 \\ \hline
\end{tabular}
\end{table}




\begin{thebibliography}{9}

\bibitem{lcoe_fraunhofer_ise}
  Levelized Cost of Electricity -- Renewable Energy Technologies,
  Christoph Kost et al. (Fraunhofer ISE),
  June 2021.
  \url{https://www.ise.fraunhofer.de/en/publications/studies/cost-of-electricity.html}
\end{thebibliography}

\end{document}
%
% ****** End of file aapmtemplate.tex ******
