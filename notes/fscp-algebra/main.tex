%% ****** Start of file aapmtemplate.tex ****** %
%%
%%   This file is part of the files in the distribution of AAPM substyles for REVTeX4.2.
%%   Version 4.2a of January 28, 2015.
%%
%
% This is a template for producing documents for use with 
% the REVTEX 4.2 document class and the AAPM substyles.
% 
% Copy this file to another name and then work on that file.
% That way, you always have this original template file to use.

\documentclass[%
 draft,
 aapm,
 mph,%
 amsmath,amssymb,
%preprint,%
 reprint,%
%author-year,%
%author-numerical,%
]{revtex4-2}

\usepackage[utf8]{inputenc}

\usepackage{amsmath}
\usepackage{amssymb}
\usepackage{mathtools}
\usepackage{upgreek}

\usepackage{siunitx}

\usepackage{mhchem}

\usepackage{xspace}

% Defining string as labels of certain blocks.

\begin{document}

% Use the \preprint command to place your local institutional report number 
% on the title page in preprint mode.
% Multiple \preprint commands are allowed.
%\preprint{}

\title{Fuel-switching carbon prices — a bit of algebra} %Title of paper

% repeat the \author .. \affiliation  etc. as needed
% \email, \thanks, \homepage, \altaffiliation all apply to the current author.
% Explanatory text should go in the []'s, 
% actual e-mail address or url should go in the {}'s for \email and \homepage.
% Please use the appropriate macro for the type of information

% \affiliation command applies to all authors since the last \affiliation command. 
% The \affiliation command should follow the other information.

\author{Philipp C. Verpoort}
%\email[]{Your e-mail address}
%\homepage[]{Your web page}
%\thanks{}
%\altaffiliation{}

% Collaboration name, if desired (requires use of superscriptaddress option in \documentclass). 
% \noaffiliation is required (may also be used with the \author command).
%\collaboration{}
%\noaffiliation

%\date{\today}

%% -----------------------
%% |      References     |
%% -----------------------
\newcommand{\secref}[2][]{Sec.~\ref{sec:#2}#1}
\newcommand{\appref}[2][]{App.~\ref{sec:#2}#1}
\newcommand{\chapref}[2][]{Chap.~\ref{chap:#2}#1}
\newcommand{\figref}[2][]{Fig.~\ref{fig:#2}#1}
\newcommand{\tabref}[2][]{Tab.~\ref{tab:#2}#1}
\newcommand{\eqnref}[2][]{Eq.~(\ref{eq:#2}#1)}

\newcommand{\subfig}[1]{{\bf(#1)}}


%% -----------------------
%% |        maths        |
%% -----------------------
\newcommand{\arr}[2]{\begin{array}{#1}#2\end{array}}
\newcommand{\op}[1]{\operatorname{#1}}
\newcommand{\punc}[1]{\,#1}
\newcommand{\anglemean}[1]{\left\langle#1\right\rangle}

\renewcommand{\d}{\ensuremath{ \mathrm{d} }}
\newcommand{\dd}[1]{\ensuremath{ \mathrm{d}#1 }}
\newcommand{\ddd}[2]{\ensuremath{ \mathrm{d}^{#2}\!{#1} }}
\newcommand{\pdb}[2]{\frac{\partial #1}{\partial #2}}
\newcommand{\dbyd}[3][{}]{\frac{\mathrm{d}^{#1}#2}{\mathrm{d}{#3}^{#1}}}
\newcommand{\dbydsecond}[3]{\frac{\mathrm{d}^2#1}{\mathrm{d}#2\,\mathrm{d}#3}}


%% -----------------------
%% |       various       |
%% -----------------------
\newcommand{\prc}{p_{\mathrm{c}}}
\newcommand{\prYX}{p_{Y\rightarrow X}}
\newcommand{\prg}{p_{r\rightarrow g}}
\newcommand{\prb}{p_{r\rightarrow b}}
\newcommand{\prbg}{p_{b\rightarrow g}}



%% -----------------------
%% |      molecules      |
%% -----------------------
\newcommand{\hydrogen}{\ce{H2}\xspace}
\newcommand{\carbdiox}{\ce{CO2}\xspace}
\newcommand{\carbmono}{\ce{CO}\xspace}
\newcommand{\ammonia}{\ce{NH3}\xspace}
\newcommand{\methane}{\ce{CH4}\xspace}
\newcommand{\methanol}{\ce{MeOH}\xspace}
\newcommand{\ethanol}{\ce{EtOH}\xspace}
\newcommand{\nitrogen}{\ce{N2}\xspace}


\maketitle %\maketitle must follow title, authors, abstract

% Body of paper goes here. Use proper sectioning commands. 
% References should be done using the \cite, \ref, and \label commands
\section{Formular for switching price}

The total cost $C_X$ of a fuel $X$ is defined as
\begin{equation}
    C_X = C^0_X + \prc * e_X \punc,
\end{equation}
where $C^0_X$ is the direct cost (independent of carbon cost) in \si{\$}, $\prc$ is the carbon price in \si{\$\per\tonne_{\ce{CO2}}}, and $e_X$ are the emissions (or carbon intensity) in \si{\tonne_{\ce{CO2}}\per\kWh}. We define the $Y$-to-$X$ fuel-switching cost $\Delta C_{YX}$ (i.e.~the cost incurred in switching from fuel $Y$ to fuel $X$) as
\begin{align}
    \Delta C_{YX} &= C_X - C_Y \\
                  &= C^0_X - C^0_Y + \prc * (e_X - e_Y) \punc.
\end{align}
The $Y$-to-$X$ fuel-switching carbon price $\prYX$ (i.e.~carbon price above which it is economically incentivised to switch from fuel $Y$ to fuel $X$) is defined as
\begin{align}
                    & \Delta C_{YX} (\prc = \prYX) = 0 \punc, \\
\intertext{and hence it is}
                    &~~~ C^0_X - C^0_Y + \prYX * (e_X - e_Y) = 0 \\
  \Leftrightarrow   &~~~ \prYX = \frac{C^0_X - C^0_Y}{e_Y - e_X} \punc.
\end{align}

\section{Green-blue switching}

We now consider the fuels $X$ and $Y$ to be one of $\{g, b, r\}$, which refer to green, blue, and reference (e.g.~natural gas). In that case we have three switching prices:
\begin{align}
    \prg    &=     \frac{C^0_g - C^0_r}{e_r - e_g} \punc,\\
    \prb    &=     \frac{C^0_b - C^0_r}{e_r - e_b} \punc,\\
    \prbg   &=     \frac{C^0_g - C^0_b}{e_b - e_g} \punc.
\end{align}
We can rewrite the equations for $\prg$ and $\prb$ as
\begin{align}
    C^0_g &= C^0_r + \prg * (e_r - e_g) \quad\text{and} \\
    C^0_b &= C^0_r + \prg * (e_r - e_b) \punc,
\end{align}
and hence it is
\begin{align}
    C^0_g - C^0_b &= \prg (e_r - e_g) - \prb (e_r - e_b) \\
                  &= \prb \, e_b - \prg \, e_g + (\prg - \prb) \, e_r \punc.
\end{align}
We can substitute that back into the equation from above for $\prbg$, which yields:
\begin{align}
    \prbg &= \frac{\prb \, e_b - \prg \, e_g + (\prg - \prb) \, e_r}{e_b - e_g} \\
    &= \frac{\prb \, e_b + \prb \, e_g - \prb \, e_g - \prg \, e_g}{e_b - e_g}\\ &~~~~+ \frac{(\prg - \prb) \, e_r}{e_b - e_g} \\
    &= \prb + (\prb - \prg) \frac{e_g}{e_b-e_g} \\ &~~~~+ (\prg - \prb) \frac{e_r}{e_b-e_g} \\
    &= \prb + (\prg - \prb) \underbrace{\frac{e_r - e_g}{e_b - e_g}}_{=: \alpha} \\
    &= \prb + \alpha * (\prg - \prb) \punc.
\end{align}

\section{Possible cases}
Let us assume that green hydrogen is always cleaner than blue and that both are cleaner than the reference (e.g.~natural gas). Hence, it is $e_g < e_b < e_r$. From $e_r > e_b \Leftrightarrow e_r - e_g > e_b - e_g$ and with $e_b < e_g$ we find that $\alpha > 1$. We therefore find the follwing cases depending on the relative values of $\prg$ and $\prb$:

\vspace{.2cm}
\noindent\textbf{Case 1: $\prg > \prb$}
\begin{align}
    \prbg &~=~ \prb + \alpha * \underbrace{(\prg - \prb)}_{>\,0} \\
            & \overset{\alpha>1}{>} \prb + \prg - \prb = \prg \punc.
\end{align}
Hence: $\prbg > \prg > \prb$.

\vspace{.2cm}
\noindent\textbf{Case 2: $\prg < \prb$}
\begin{align}
    \prbg &~=~ \prb + \alpha * \underbrace{(\prg - \prb)}_{<\,0} \\
            & \overset{\alpha>1}{<} \prb + \prg - \prb = \prg \punc.
\end{align}
Hence: $\prbg < \prg < \prb$.

\vspace{.2cm}
\noindent\textbf{Case 3: $\prg = \prb$}
\begin{align}
    \prbg &~=~ \prb + \alpha * \underbrace{(\prg - \prb)}_{=\,0} = \prb\punc.
\end{align}
Hence: $\prbg = \prb = \prg$.



\end{document}
%
% ****** End of file aapmtemplate.tex ******
